%----------------------------------------------------------------------------------------------------------------------------------------------------
% INTRODUCTION
\section{Introduction}
\IEEEPARstart{D}ay by day power system is becoming more and more complex, which makes it impossible to operate it without automation for higher reliability. Due to wider geographical distribution of the grid, timely detection of faults and taking preventive/corrective countermeasure  has become a complex task, which requires a reliable, fast-acting and absolute technique to deal with the challenge. This is where the Phasor Measurements Unit (PMU) comes into the picture. 

A phasor is a complex number which represents both magnitude and angle of an AC quantity. And the synchronized sampling/ measurement of this phasor at a precise reference (time) is called \textbf{synchrophasors} \cite{std:c37}. Using these synchrophasor measurements, different quantities are derived like phase angle, frequency, rate of change of frequency (ROCOF) etc. Frequency is computed as the first derivative of the synchrophasor phase angle and ROCOF is computed as the second derivative of the same phase angle. 

Just like all other engineering devices PMU's reliability, accuracy  and precision are very crucial for its application and hence different kinds of test are done to validate its performance. Hence just like other measuring devices PMU standards are defined which states minimum performance requirement(s). All device should at least meet the requirement stated by the standards, according their application. 

So, here with this project we are exploring the application of miniature Full Spectrum Simulator as a Hardware -in-loop- testing device for PMUs and the implementation of standards on indigenously developed PMU . We will see the feasibility as well as effectiveness of the platform and how it can be adjusted to make it suitable for C37.118 compliance testing. This report is largely divided in two parts via three sections 1) C37.118 standard 2) Tests and its explanation and 3) Implementation. The \textit{compliance} section provides theory background of the topic and basis of tests. Implementation section provides insight about the intended test-setup, hardware detail and the approach taken. 

\subsection{Background Theory}
Phasor representation of sinusoidal signals is commonly used in AC power system analysis. The sinusoidal waveform defined in Equation (1):

\begin{equation}
x(t) = X_m(t) cos(\omega_{0} t + \phi)
\end{equation}
is commonly represented as the phasor as shown in Equation (2):

\begin{equation}
\textbf{X} = \frac{Xm}{\sqrt{2}}\exp^{j\phi(t)}
or\hspace{2pt} \textbf{X} = X_r+jX_i
\end{equation}

The \textit{synchrophasor} representation of the signal $x(t)$ in (1) is the value X in (2) where $\varphi$ is the instantaneous phase angle relative to a cosine function at the nominal system frequency synchronized to UTC. Before we go into the compliance theory we will go through the basic definitions for clarity:\\
\textbf{Phasor:} A complex equivalent of a sinusoidal wave quantity such that the complex modulus is the cosine wave amplitude, and the complex angle (in polar form) is the cosine wave phase angle.\\
\textbf{UTC:} Its is the time of day at the earth's prime meridian.\\
\textbf{ROCOF:} It is the measure at which the frequency changes in a given instance of time.\\
\textbf{Rate of change of Frequency Error (RFE):} The measure of error between the theoretical ROCOF and the measured ROCOF for the given instant of time.\\
\textbf{Frame:} A data frame or a frame of data is a set of synchrophasor, frequency, and ROCOF measurements that corresponds to the same time stamp.\\