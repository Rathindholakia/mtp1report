%----------------------------------------------------------------------------------------------------------------------------------------------------
% INTRODUCTION
\section{Standard Compliance}
\IEEEPARstart{E}very PMU should be able to caluculate the value of phasor estimate accurately. The estimate will include positive sequence or single phase values, phase difference, freqnency and ROCOF. So it is importnat to keep in mind that the measurements are actually estimates of certain values.

Now, for a given input wave the computation for estimating the desired quantity are given below:
For estimating frequency:
\begin{equation}
f(t) = \frac{1}{2\pi} \dfrac{d\psi(t)}{dt}
\end{equation}
The ROCOF is defined as:
\begin{equation}
ROCOF(t) = \dfrac{df(t)}{dt}
\end{equation}
important thing to note here is that phasors are always computed in relation to the system nominal frequency ($f_0$). Here $\psi(t) = \omega_0 t + \varphi(t)$

\subsection{Measurement Evaluation}
To validate the estimation coming from PMUs they are compared with the theoretical results. As results consists of amplitude and phase difference both they are considered combinedly and this quantity is called \textit{total vector error} (TVE). TVE is an expression of difference between "perfect" sample of a theoretical synchrophasor and the estimate given by the unite at the same instant of time \cite{std:c37}. The value is normalized and expressed in PU of the theoretical phasor:
\begin{equation}
PVE(n) = \sqrt{\frac{\hat{X}_r(n) - \hat{X}_r(n)} {den}}
\end{equation}
to be complient with standard,PMU shall provide synchrophasor, frequency, and ROCOF measurements that meet the requirements as per the standards.


