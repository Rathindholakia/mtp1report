\documentclass{beamer}
\usetheme[everytitleformat=regular]{m}              % Use metropolis theme
\usepackage{booktabs}
\usepackage{amssymb}
\usepackage{amsmath}
\usepackage[scale=2]{ccicons}
\usepackage{graphicx}
\usepackage{color}
\usepackage{transparent}
\usepackage{animate}


\title{Hardware-In-the-Loop Testing of Phasor Measurement Unit using Mini-Full Spectrum Simulator}
\date{\today}
\author{Rathin Dholakia}
\institute{MTP Stage - 1}
\begin{document}
  \maketitle
  \section{Introduction}


 %=============  Slide - 2
  \begin{frame}{Overview}
 	\begin{enumerate}
  	\item Theoretical background
  	\item Literature Survey
  	\item Proposed Scheme
  	\item Work Done
  	\item Plan of work
 	\end{enumerate}
   \end{frame}
 %============Slide - 2 ends ===========
 
 
   %---- SLIDE - 3  ================
     \begin{frame}{Why Precise Time?}
   	 \begin{center}
		\begin{itemize}
		\item Event Reconstruction - 14, Aug, 2003 North America 
		\item Synchro Phasors / WAMS - Geographically separated
		\item Multi-rate Billing
		\item Traveling-wave Fault Detection 
		\item Testing and verification of protective devices
		\end{itemize}
   	 \end{center}
     \end{frame}
    %============= Slide - 3 END ==============
 
   %---- Slide - 4 ---------------
   \begin{frame}{What - Basic Requirements}
	 \begin{itemize}
	 	\item Reference synchronized to national standard - UTC, Atomic clock
	 	\item \textbf{Robustness:} Survive contingency, unstable clock
	 	\item Should be \textbf{automated}, \textbf{customizable} and \textbf{efficient}
	 	\item Not cause system infrastructure overhead.
	 	\item \textbf{Security:} immune to intrusion \& withstand potential attacks
	 \end{itemize}
   
   \end{frame}
   %---- Slide - 4 ENDs --------------- 

   %---------Slide-4 Starts ----------------
   \begin{frame}{How - can we get it?}
   There are different sources of accurate time:
     \begin{enumerate}
     	\item \textbf{Atomic clock}: Ultimate reference, cesium beam oscillator. Costly but longer stable operation  
     	\item \textbf{GPS:} US DoD, Started 1980s - NAVSTAR, 1993- full functional, 24 satellites 
     	\item \textbf{GLONASS:} 1976 USSR, better than GPS
     	\item \textbf{Standard Radio Transmission:} Oldest time-keeping method, existent since 1920 
     	\item Microwave or Terrestrial Distribution: US, UK \& Germany, sensitive to corona
     \end{enumerate}
   \end{frame}
   %---------Slide-4 Ends ----------------

   %---------Slide-5 Starts ----------------
   \begin{frame}{How - can we get it? \tiny{(continued)}}
   	Methods of Time Distribution:
   	\begin{enumerate}
   		\item Dedicated Timing Signals:
   		\begin{itemize}
   			\item 1 Pulse Per Minute (1 PPS):
   			\begin{itemize}
   				\item[-] Excellent accuracy $100 ns$
   				\item[-] Simple, easy, flexible
   				\item[-] Ambiguity of 1 year
   			\end{itemize}
   			\item IRIG 
   			\begin{itemize}
   				\item[-]  Inter-Range Instrumentation Group [1956-1970]
   				\item[-] 7 codes [A - H] \& 2 Types [Modulated - Unmodulated]
   				\item[-] Accuracy range: $10-50 ns$ 
   				\item[-] Currently MOST used method commercially 
   			\end{itemize}
   		\end{itemize}
   	\end{enumerate}
   	\tiny continued...
   \end{frame}
   %---------Slide-5 Ends ----------------


   %---------Slide-6 Starts ----------------
   \begin{frame}{How - can we get it? \tiny{(continued)}}
   	Engineering is about: Feasible - Simple - Cheap - Reliable solutions
   	\small \textit{``Can't we use the same network to timekeep \& send data?"}\\

   	\begin{enumerate}
   		\setcounter{enumi}{1}
   		\item   Network Time Synchronization:
   		\begin{itemize}
   			\item NTP/SNTP:
   			\begin{itemize}
   				\item[--] Existing since 1980s
   				\item[--] Reliable, flexible, simple, cheap, robust 
   				\item[--] Accuracy range: $10 - 100~ms$
   				\item[--] Most widely used
   			\end{itemize}
   			\item PTP/IEEE 1588:
   			\begin{itemize}
   				\item [--] Extremely accurate
   				\item [--] 2 modes : Software only \& Hardware-aided
   				\item [--] Accuracy range: $20 - 100~ns$
   				\item [--] Slowly replacing NTP, 1 PPS \& IRIG-B
   			\end{itemize}
   		\end{itemize}	
   	\end{enumerate}
   \end{frame}
   %---------Slide-6 Ends ----------------
   \section{Detailed Study}
      %---------Slide-7 Starts ----------------
      \begin{frame}{Network Time Protocol}
      	\begin{enumerate}
      		\item Epoch started on 1 Jan, 1972 at  2,272,060,800.0 seconds.
      		\item 64-bit word length.
      		\item Stratum, Synchronization Distance \& Dispersion.
      	\end{enumerate}
     \begin{figure}[!t]
     	\centering 
%     	\def\svgwidth{150pt} 
%     	\includegraphics[scale=0.45]{ntp_delay.png}
 %    	\caption{NTP Delay Calculation} 
 %    	\label{fig:ntp_delay} 
     \end{figure}
    \centering $a = T_{i-2}- T_{i-3}$ \quad and \quad $b =T_{i-1}- T_{i}$    
	\begin{displaymath}
	\centering
	\delta_{i} = a - b \quad \textrm{and} \quad \theta_{i} = \frac{a+b}{2} 
	\end{displaymath}      	 
      \end{frame}
      %---------Slide-7 Ends ----------------

   %---------Slide-8 Starts ----------------
   \begin{frame}{Contingency}
   	\begin{figure}
  % 		\includegraphics[scale=0.3]{ntp_ntwrk.png} \quad
   %		\includegraphics[scale=0.3]{ntp_ntwrk2.png}
   	\end{figure}
   	In case of network disruption, NTP adapts itself to optimum configuration, it uses algorithms like (but not limited to): 
   	\begin{itemize}
   		\item Peer-selection algorithm
   		\item Convergence \& Consistency algorithm, minimum filter \& minimum average algorithm.
   		\item Agreement algorithm for peer-selection (adaptation of maximum likelihood algorithm )
   	\end{itemize} 
   \end{frame}
   %---------Slide-8 Ends ----------------

   %---------Slide-9 Starts ----------------
   \begin{frame}{PTP/ IEEE 1588}
   	\begin{itemize}
   		\item Defined in 2002, standard quotes:\\
   		 {\small \textit{``IEEE 1588 is designed to fill a niche \alert{not well served} by either of the two dominant protocols, NTP and GPS. IEEE 1588 is designed for \alert{local systems} requiring accuracies beyond those attainable using NTP... for applications that \alert{cannot bare cost of GPS} receivers.." }}
   		\item {\normalsize Two modes} - Software-only \& Hardware assisted
   		\item It can be called an enhanced version of NTP.
   		\item Hardware-aided mode - highly accurate: 1 - 10 ns
   		\item CERN is using enhanced PTP with accuracy of $3~sps$ 
   	\end{itemize}

   \end{frame}
   %---------Slide-9 Ends ----------------
   
   %---------Slide-10 Starts ----------------
    \begin{frame}{How is it different than?}
    	\begin{columns}
    		\column{0.4\textwidth}
   		
    		\column{0.6\textwidth}

    		\begin{itemize}
    			\item Calculation same as NTP but
    			\item Variable path delay calculated in hardware mode:
    			\begin{itemize}
    				\item[--] Transparent Clock 
 	    			\item[--] Boundry clock 
 	    			\item[--] Grandmaster clock
    			\end{itemize} 
    			\item Only one reference (1 grandmaster at a time)
    			\item Proactive delay synching 
    		\end{itemize}
    	\end{columns}
   
    \end{frame}
    %---------Slide-10 Ends ----------------
      
      \section{Case Study}
    %---------Slide-11 Starts ----------------
    \begin{frame}{NTP v/s PTP}
blahblahsjhkjsdkjsd
   \end{frame}
    %---------Slide-11 Ends ----------------

    %---------Slide-11 Starts ----------------
    \begin{frame}{NTP v/s PTP \small(Case-1, link-1 fails)}
adslsdflsdf
    \end{frame}
    %---------Slide-11 Ends ----------------

    %---------Slide-11 Starts ----------------
    \begin{frame}{NTP v/s PTP \small(Case-2, link-3 fails)}

fake fake
    \end{frame}
    %---------Slide-11 Ends ----------------

    %---------Slide-11 Starts ----------------
    \begin{frame}{NTP v/s PTP \small(Case-3, link-2 fails)}
fakealkds;lksj;oiwejr
    \end{frame}
    %---------Slide-11 Ends ---------------

    %---------Slide-11 Starts ----------------
    \begin{frame}{NTP v/s PTP \small(Case-4, Both masters fails)}
werwerqwerq
    \end{frame}
    %---------Slide-11 Ends ---------------
    %---------Slide-11 Starts ----------------
    \begin{frame}{NTP v/s PTP\small why PTP is preferred}
werwesdfsdfs
    \end{frame}
    %---------Slide-11 Ends ---------------

    %---------Slide-11 Starts ----------------
    \begin{frame}{Conclusion}
\begin{columns}
	\column{0.5\textwidth}
	\textbf{NTP}
	\begin{itemize}
		\item Can work on network of any dimension
		\item No additional hardware
		\item No network modification
		\item Practically no additional cost
		\item Least possible network overhead
		\item Simple - cheap - robust - flexible
	\end{itemize}
	
	\column{0.5\textwidth}
	\textbf{PTP}
	\begin{itemize}
		\item  Highly Accurate 
		\item Variable packet delay compensated
		\item  Can handle greater network disruption - Robust
		\item Reliability is high so mission critical
	\end{itemize}
	\vspace{1.7cm}
\end{columns}
    \end{frame}
    %---------Slide-11 Ends ---------------

    
    %---------Slide-11 Starts ----------------
    \begin{frame}{References}
\begin{thebibliography}{1}
	\tiny
	%\bibitem{IEEEhowto:kopka}
	%H.~Kopka and P.~W. Daly, \emph{A Guide to \LaTeX}, 3rd~ed.\hskip 1em plus
	%  0.5em minus 0.4em\relax Harlow, England: Addison-Wesley, 1999.
	
	\bibitem{NTP}
	David L.  Mills,  Member,  IEEE \emph{Internet  Time  Synchronization: 
		The  Network  Time  Protocol }, IEEE  TRANS. COMM., Vol.~39, No.10,  1991 
	\bibitem{PTP}
	IEEE Std 1588-2008: Technical Committee on Sensor Technology (TC-9), Revision of 2002.
	
	\bibitem{theFutureofTimeSynch}
	B.~Baumgartner, C.~Riesch and Manfred Rudigier, \emph{The Future of Time Synchronization in the Electric Power Industry}, OMICRON electronics GmbH.
	
	\bibitem{NTPandPTP}
	Precise Time and Frequency, Inc., \emph{NTP and PTP}, Precise Time and Frequency, Inc., USA  
	
	\bibitem{TimeinPower}
	Bill Dickerson, \emph{Time in the Power Industry}, Arbiter Systems Inc.,
	
	\bibitem{NTPwiki}
	\url{https://en.wikipedia.org/wiki/Network_Time_Protocol}
	
	\bibitem{NTPsite}
	\url{ntp.org}
	
	\bibitem{NTPvsPTP}
	\url{http://www.en4tel.com/pdfs/NTPandPTP-A-Brief-Comparison.pdf}
	
	\bibitem{comparePTPNTP}
	\url{http://blog.meinbergglobal.com/2013/11/22/ntp-vs-ptp-network-timing-smackdown/}
	
	\bibitem{RFC958}
	David Mills, \emph{RFC 958}, \url{https://tools.ietf.org/html/rfc958}
\end{thebibliography}
    \end{frame}
    %---------Slide-11 Ends ----------------


   %---------Slide-4 Starts ----------------
   \begin{frame}
   \begin{center}
   \includegraphics[scale=0.3]{thanku.png}
   \end{center}
   \end{frame}
      %---------Slide-4 Ends ---------------- 
\end{document}