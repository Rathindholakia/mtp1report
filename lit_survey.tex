\section{Literature Survey}

Apart from studying IEEE C37.118 for the purpose of tests and the standardize condition, a literature survey was also done. Papers refering to the topics like Dynamic testing, IEEE standards were taken into consideration for a broader view and dipper understanding. Herewith presenting a brief discussion on selected papers:
One of the goal of this experiment is to \textit{access the feasibility} of FSS as a HIL testing platform for PMU so in that context a very similar paper was found ``\textbf{Development of a Smart Grid Test Bed and Applications in PMU and PDC Testing}"  which discusses the development of "smart-grid" testbed. In the given paper the activity is carried out using RTDC as a platform. This paper tries to present the scenario of how one can model a smart power system within laboratory and how that setup can be used for testing of synchrophasors and phasor data concentrators. The \textit{test setup} consisted of RTDC by Opal-RT which houses the power system model, aroudn which, system was created which had a SEL GPS module, which provided time reference to whole setup, an array of different relays having different charactersitics in conjunction of communication device. The output of RTDS is given to other relays and the PMU under test via current and voltage amplifiers. which boosts the low voltage signals to normal power level. For the purpose of reliable testing, PMU was coupled with RTDS as well as PMU TESTER POM2-6143\cite{Paper:saugata}. PMU tester has accuracy of 0.01 \% as per the reference. Tests performed on PMU were of 4 kinds 1) Balance Condition 2) Unbalance Condition 3) System at Off-nominal Frequency and 4) System with harmonics component. Each test was first conducted using PMU Tester, readings were taken and then it was repeated using RTDS. Both of these results were than compared. Apart from testing PMU this test-bed also evaluated the capabilities of PDC, this was possible due to embedded PDC and data logging facilities in SEL$\circledR$ relays, which were connected through PC. The characteristics evalueted were 1) Data Validation 2) Data Re-Sampling 3) Data Alignment 4) Data Recognition 5) Data Retrieval 6) Data Truncation.

Another interesting paper studied was \textbf{Dynamic PMU Compliance Test under C37.118.1aTM-2014} by R. Ghiga, this paper present flexible testing methodology for dynamic compliance test for PMUs \cite{Paper:ghiga}. Three different PMUs are taken and tests are conducted on them using laboratory hardware setup and comparative study is presented. This setup used Doble F6150 Power System Simulator, which is used for relatesting, this device acts in standalone mode and generates 3 -$\phi$ AC voltage and current signals with varying amplitude and frequency. PMUs can be connected directly, without the amplication stage or so, which resolves the accuracy concerns and in addition more than 1 device can be connected at a time, which enables testing of all 3 devices at a time! Tests were conducted on the basis of the PMU standard which are Amplitude Modulation, phase modulation, freq ramp, amplitude step and phase step. The results presented were quite interesting. An important note to make over here is that only dynamic tests are performed and the test signals given to the PMUs are not time triggered. As per the C37.118-2011 the step-input test signals requires to be UTC synchronised. So this implementation is considred to quite simlar to ours.